% Top-level PhD Dissertation file

% Generously contributed by Rance Necaise
%                    PhD, August 1998
%                    Topic:  IMPROVEMENTS TO THE COLOR QUANTIZATION PROCESS
% Published and maintained by Professor William L. Bynum
%                             http://www.cs.wm.edu/~bynum/

% Additional changes by Bob Matthews == rem
%                        PhD
% to conform to Graduate Arts and Sciences Thesis guide of Nov. 2003.
% 

% Further changes done by Ruth Lambrecht
%                    PhD, May 2013
%                    Topic:  Translating Spatial Problems into Lumpable Markov Chains
% with help from Andrew Pyles
%                    PhD, May 2013
%                    Topic:  Network Traffic Aware Smartphone Energy Savings
% to conform to the standards set on 10/08/12.
% 

% Modifications to comply with Physical Standards set on 08/13/2015 done by David T. Nguyen
%            PhD, February 2016 
%            Topic: Enhancing Mobile Device System Using Information from Users and Upper Layers
% Compiling in Ubuntu: use Kile as an editor, and use XeLaTeX button to compile
% Need to instal MS fonts first as follows
%               sudo apt-get install ttf-mscorefonts-installer
%               sudo fc-cache
% After that, check with 
%               fc-match Arial
% Use PDF figures!!! (for some reason EPS figures are not displayed correctly, 
% you can use 'epspdf myfigure.eps' to convert)
%

% Modifications made by Ed Novak
%            PhD, June 2016 
%            Topic: Security And Privacy For Ubiquitous Mobile Devices
% in April 2016 to make the template compile on department machines, via command 
% line (xelatex) and to be easier to use / understand.
%

% At the time (May 2018), please refer 
% http://www.wm.edu/as/graduate/studentresources/thesis-dissertations/physicalstandards/index.php
% for the latest standard.



\documentclass[11pt,draft]{wmthesis}

% Options
% -------
% proposal - if you are writing a proposal, e.g., \documentclass[11pt, proposal]{wmthesis}, no approval page, acknowledge, dedication pages.
% draft - if you are writing a draft, including everything (blank approval page) as ready to be signed
% final - if you are writing a final, will replace approval page with signed_approval_page.pdf
% thesis - if you are a master student, add thesis option, e.g.,  \documentclass[11pt, draft, thesis]{wmthesis}
% dissertation - default

% Refine the toc styple to the latest standard - Shanhe
% \usepackage[titles]{tocloft}
% \newlength\mylength
% \renewcommand\cftchappresnum{\chaptername~}
% \renewcommand\cftchapaftersnum{.}
% \renewcommand{\cftdot}{}
% \settowidth\mylength{\cftchappresnum\cftchapaftersnum\quad}
% \addtolength\cftchapnumwidth{\mylength}


%%----------------------------------------------------------------
% Set packages 
%%----------------------------------------------------------------
\usepackage{lipsum} % for dummy text
\usepackage{todonotes} % for dummy image


% Some very useful LaTeX packages include:
% (uncomment the ones you want to load)
% \usepackage{graphicx}
% \usepackage{color}
% \usepackage{url}
% \usepackage{epsfig}
% \usepackage{epstopdf}
% \usepackage{verbatim}



%%----------------------------------------------------------------
% Set font 
%%----------------------------------------------------------------
\ifxetexorluatex
    \usepackage{fontspec} % only use it with XeLaTeX or LuaLaTeX, recommended. -- Shanhe
    \setmainfont{Arial}
\else
    % %%-- This is how to set Arial font using LaTeX or PdfLaTeX, not recommended. -- Shanhe
    % %% If you don't install the font into tex packages, then you will just get Computer Modern font.
    % %% How to install uarial package in Mac OS (required MacTeX, /Library/TeX/texbin in your path). *nix with TexLive should be similar. - Shanhe
    % %%   1. curl --remote-name https://www.tug.org/fonts/getnonfreefonts/install-getnonfreefonts
    % %%   2. sudo texlua install-getnonfreefonts
    % %%   3. sudo getnonfreefonts --sys -a
    % %% Then uncommen the following two lines.
    % \usepackage{uarial}
    % \renewcommand{\familydefault}{\sfdefault}
\fi


%%----------------------------------------------------------------
% Set indent 
%%----------------------------------------------------------------
% The wmthesis class is based on the latex report class which
% only indents paragraphs if they immediately follow other paragraphs.  The
% dissertation lady says this is wrong.  I tend to give more credence
% to Dr. Knuth (author of TeX) on this issue, since the other way looks really
% crappy.  If you want the first line of every paragraph indented,
% uncomment the next line to include the indentfirst package. -- rem
% Not sure if this is still an option -- Ruth
% Not an issue any more, but I keep this line in case -- Shanhe
% \usepackage{indentfirst}





\begin{document}
\doublespacing

%%--Set thesis metadata


% Provide the full title of your thesis or dissertation using the format for upper and lower case as indicated.
\thesisTitle{Dissertation Title}

%  \thesisAuthor macro
%     defines two TeX variables (only usable in this file)
%  \thesis@author  is assumed to be a "short" version of the author's name
%        used on the title page
%  \thesis@authorx is assumed to be the full name of the author
%        used on the approval, the UMI abstract, and the vita pages
%  For example
%     \thesisAuthor{A. Goode Student}
%        sets both \thesis@author and \ thesis@authorx to
%        "A. Goode Student"
%     \thesisAuthor[Aloysius Goode Student]{A. Goode Student}
%        sets \thesis@author  to "A. Goode Student" and
%             \thesis@authorx to "Aloysius Goode Student"
\thesisAuthor[Student Name]{Student Name}

% Enter the expected conferment month and year of your degree, [e.g. January, May, August]. 
% Check with the Office of Graduate Studies and Research to verify the actual graduation month and year.
\thesisMonth{May}
\thesisYear{2018}
\thesisAdvisor{Professor Qun Li} % Advisor name with title

% Enter Degree, e.g. Master of Science or Master of Arts or Doctor of Philosophy.
\thesisType{Doctor of Philosophy}


% "Provide your hometown and state in the following format [e.g. St. Louis, Missouri]. 
% International students should enter their hometown, state/province, and country [e.g. Montreal, Quebec, Canada]"
% \thesisLocation{Xiangyang, Hubei, China} % @deprecated  this location should be hometown, using thesisHometown{} instead
\thesisHometown{town/city, state}

%%-- Degrees earned previous to Ph.D.
%% note that the degree should be spelled out, not abbreviated
% "List all previous degrees with the most recent degree first,
% [e.g. Master of Arts,University of Colorado-Boulder, 1987]."
\thesisDegreeOne{Master of Science, ABC College, 2013}
\thesisDegreeTwo{Bachelor of Engineer, ABC College, 2010}

% Enter your department name, [e.g. Department of History, Department of Biology, or American Studies Program].
\thesisDepartment{Department of Computer Science}

%%-- Committee members
%% example 
%%  \thesisCommittee[Department]{Professor John Doe}{ABC College}
%%  \thesisCommittee{John Doe}{XYZ Company}
%%  \thesisCommittee[Copration Department]{John Doe}{XYZ Company}
\thesisCommittee[Computer Science]{Professor A}{College of William \& Mary}  % use {College of William \& Mary} -- Shanhe
\thesisCommittee[Computer Science]{Associate Professor B}{College of William \& Mary}
\thesisCommittee[Computer Science]{Assitant Professor C}{College of William \& Mary}
\thesisCommittee[Electrical and Computer Engineering]{Professor D}{College of William \& Mary}
\thesisCommittee[Applied Science]{Outside Committee Member}{College of William \& Mary}


%%-- Insert contents of abstract.tex, acknowledge.tex and the dedication.  Don't
%%forget to check these files for formatting hints.
%% Also, the order they are given right here does not matter
\thesisAbstract{abstract.tex}

% In the actual finished pdf document the TOC comes next
\thesisAcknowledge{acknowledge.tex}

\thesisDedication{I would like to dedicate this dissertation to my parents, Jane and John Doe who provided endless support and love throughout my time at William and Mary.}



%%--Create the thesis Prolog
\makeProlog


%%-- contents of the actual thesis feel free to \input as many files as you want
\chapter{Introduction}
This is where the content of your dissertation goes.  You can directly include chapters like the introduction, related work, conclusion, and future work in this file.  You can also call ''\textbackslash input\{\}'' in this file to include your other papers and .tex documents easily.  When looking at the raw .tex version of this section, you can see how to use input and separate chapters as a comment below.

Here I will make a citation: \cite{mobile_marketshare}.  You must have at least one citation for ''\$make all'' to finish successfully!  Also, please note the type of quotations used.

% Actual paper / chapter 1
%\input{chapter_one_content.tex}
%\setcounter{equation}{0}
%\cleardoublepage


\chapter{proj1}
\label{cht:proj1}
\subimport*{proj1/}{main}
\setcounter{equation}{0}
\cleardoublepage


\chapter{proj2}
\label{cht:proj2}
\subimport*{proj2/}{main}
\setcounter{equation}{0}
\cleardoublepage


\chapter{proj3}
\label{cht:proj3}
\subimport*{proj3/}{main}
\setcounter{equation}{0}
\cleardoublepage


\chapter{Conclusion and Future Work}
\label{cht:proj3}
\section{Future Work}

\lipsum[2-2]

\section{Conclusion}

\lipsum[4]



%%-- If you want to add some appendices uncomment \appendix below
%% All this actually does is start calling the "chapters" "appendices"
%\appendix
%\input{appendixA}
%\input{appendixB} ...


%%-- List of references not actually cited in the document (\nocite's)
%% \nocite{NDSS04DTLS}


%%--Include the bibliography
\makeThesisBib{thesis}


%% Vita is optional
%\makeThesisVita{vita}


\end{document}
